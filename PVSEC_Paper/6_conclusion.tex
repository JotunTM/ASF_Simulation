% !TEX root = 99_main.tex

In this paper we present a simulation methodology to evaluate a dynamic photovoltaic shading system, combining both electricity generation, and the energy demand on the building. The methodology can be applied to evaluate different PV systems, building systems, building typologies and climates. It is then coupled with a post processing python script to determine the optimum system configuration for control. 

The dynamic PV integrated shading system has clear advantages to a static system as it can adapt itself to the external environmental conditions. This enables it to orientate itself to the most energy efficient position. The resulting choice of an open or closed configuration is sensitive to the building system and location. The use of LED lights, for example, reduces the weighting of the lighting energy demand. This would result in closed configurations optimised for cooling to over-ride the open positions. 

This work ultimately presents a methodology for the planning and optimisation of sophisticated adaptive BIPV systems. Future work will use this methodology to determine the environments and building typologies that could benefit from adaptive BIPV systems. 

%Moved to Introduction: The work presented in this paper is applied in the context of the Adaptive Solar Façade (ASF) project. The ASF is a lightweight PV shading system that can be easily installed on any surface of new or existing buildings. The ASF consists of a modular frame and pneumatically actuated panels to control glare and solar gain, as well as for two-axis PV tracking. It has been implemented at the ETH House of Natural Resources will be installed at the NEST HiLo building at EMPA (www.hilo.arch.ethz.ch). 

% \section{Outlook}
% \label{ch:outlook}

% [To be Decided]