%!TEX root = thesis.tex
\chapter{Introduction}

%(mention that this PhD is within a larger research portfolio, this could also be moved to Objectives)

\section{Motivation and Literature Review}\
\label{ch:motivation}
	%(Context and justification, why is this work important?)\\
	Buildings are at the heart of society and currently account for 32\% of global final energy consumption and 19\% of energy related greenhouse gas emissions \cite{IPCC}. Nevertheless, the building sector has a 50-90\% emission reduction potential using existing technologies \cite{IPCC}. Within this strategy, building integrated photovoltaics (BIPV) have the potential of providing a substantial segment of a building's energy needs \cite{defaix2012technical}. Even the photovoltaic (PV) industry has identified BIPV as one of the four key factors for the future success of PV \cite{raugei2009life}. 

	%Recent developments regarding efficiency and costs of thin film BIPV technologies, in particular, CIGS, have brought new design possibilities \cite{NREL} \cite{kushiya2014cis} \cite{kaelin2004low} \cite{jelle2012building}. Their lightweight nature and customisable shapes allow for easier and more aesthetically pleasing integration into the building envelope. In addition, less power is required to actuate them, thus facilitating the development of dynamic envelope elements \cite{rossi2012adaptive}. \\


	Dynamic building envelopes have gained interest in recent years because they can save energy by controlling direct and indirect radiation into the building, while still responding to the desires of the user \cite{loonen2013climate}. This mediation of solar insolation offers a reduction in heating / cooling loads and an improvement of daylight distribution \cite{rossi2012adaptive}. Interestingly, the mechanics that actuate dynamic envelopes couples seamlessly with the mechanics required for facade integrated PV solar tracking. \cite{loonen16} reviews current building perfomance simulations of adaptive facades and points out the lack of adaptability in building simulation tools. Single axis dynamic shading has been evaluated in \cite{nielsen2011quantifying}, emphasizing the importance of numerical evaluations in facade design decisions. 

	Previous BIPV research analyses electricity production and building energy demand for static BIPV shading systems \cite{sun2010, sun2012, David2011, mandalaki2012assessment, Mandalaki2014complexPV, mandalaki2014assessment, yoo2011available, jayathissa2015abs}. The performance of fixed PV shading devices in dependence of different angles is analysed for cooling and electrical performance in \cite{sun2010} with a simplified PV electricity model. That work is extended in \cite{sun2012} to include different building orientations. In \cite{David2011} the efficiency of fixed PV-shading devices is analysed, suggesting indices for comparison. \cite{mandalaki2012assessment} concludes that fixed surrounding PV shading devices are most efficient. The same authors asses different PV simulation models in \cite{Mandalaki2014complexPV} and are able to show that extended electrical modelling is needed for complex PV geometry. \cite{mandalaki2014assessment} also includes visual comfort and finds brise-soleil systems to perform best. A first approach on assessing building energy demand with dynamic shading in combination with estimated PV electricity production is given in \cite{jayathissa2015abs}. 

	PV electricity production of shading devices has been evaluated for fixed angles in \cite{freitas2015maximizing}, where different BIPV facade geometries are analysed, finding horizontal louvers to perform best. In \cite{hofer2015PVSEC} a in-depth analysis of dynamic shading modules was evaluated for various design parameters with solar tracking. 

	This thesis expands on the work in \cite{jayathissa2015abs} and \cite{hofer2015PVSEC} by analysing dynamic PV shading systems, while also taking into account mutual shading amongst modules and its effect on PV electricity generation. The approach allows us to reduce efficiency degradation due to partial shading of PV modules \cite{hofer2015PVSEC}.

	The work presented in this thesis is applied in the context of the Adaptive Solar Facade (ASF) project \cite{nagy2015frontiers}. The ASF is a lightweight PV shading system composed of CIGS panels, that can be easily installed on any surface of new or existing buildings. This thesis will present a methodology of simulating an ASF while simultaneously calculating the energy demand of the office space behind the facade.


\section{Problem Statement}\
	%What is the problem that you are trying to solve?
	The optimization problem has to be solved for pv modules as adaptive building shading systems:\\
	\begin{equation}
			minimize(C+H+L-PV)
	      	\label{e:minimize}
	\end{equation}



\section{Objectives of Research}\
	%(What do you want to achieve, more in general terms? Bullet point list)\\

	Based on the problem statement, the objectives are to

	\begin{itemize}
		\item Develop a framework for modelling an adaptive solar facade
		\item Find the best configurations to minimize the building energy demand
	\end{itemize}


\section{Thesis Outline}\
	%Breakdown of your thesis (Not always necessary)

