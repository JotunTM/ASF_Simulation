%!TEX root = thesis.tex

\chapter{Literature Review}


General Introduction/ methods topics 

	- Photovoltaic
	- Energy Balance (U-Value, g-Value...)
	- User Comfort
	- Self Shading of panels (if evaluated)\\




The Adaptive Solar Facade: From Concept to Prototypes (Nagy15)\\


	Control
	
	- two axis solar tracking (experiment corresponds to forecast)
	- interaction with environment (user needs and interaction as well as energy savings must be taken into account)
	- voice recognition (robustness needs to be improved, redundant system is wished for by users)
	
	
	Numerical Analysis
	
	Energy Demand Savings
	- Single room
	- Three bands
	- hourly time steps over one year with 324 possible configurations analyzed in Energy Plus
	- Energy savings mainly through decreased cooling demand but also less heating and lighting than with regular blinds
	
	PV System Design
	- Takes into account: building geometry, horizon contour line, module geometry
	- determine: shadow shape, orientation towards the sun, sky view factor, visibility of reflecting elements
	- Cells should be connected in strings with similar shading to minimize electrical mismatch losses
	
	
	Prototypes
	- ETH House of Natural Resources: monitor temperature, humidity and illuminance inside ASF office as well as adjacent one
	- HiLo will have two facades, user interaction in residential building will be studied\\
	
	
	
Numerical Simulation of Energy Performance, and Construction of the Adaptive Solar Facade\\

	
	Numerical Simulations
	
	- ETH House of Natural Resources	
	- net energy demand for heating, cooling and lighting
	- heating with heatpump, cooling with district water, LED lighting
	- convert energy demand to the end electricity demand of the HoNR office
	- cooling energy predominant in summer afternoons, heating in winter during days and lighting in the evenings and mornings
	- largest absolute savings compared to no facade and standard louvers system at 45 degrees  caused by decrease of cooling demand
	- 25\% savings compared to louvers and 56 \% compared to no shading (energy demand)
	- ASF saves electricity compared to louvers but probably building needs more electricity than without shading (as heating uses the most electricity and cooling is very efficient in the chosen configuration, where electricity is only needed to pump the fresh water into the cooling system).
	- ASF produces more electricity than needed by building\\
	


Climate adaptive building shells: State-of-the-art and future challenges \cite{loonen13}\\


	-  active and passive building technology (active means utilizing modern technology (such as PV), whereas passive directly uses wind and sun to optimize the building) -> combine them
	- climate adaptive building shells (CABS) can take advantage of variable outdoor conditions
	- definition of CABS: ``A climate adaptive building shell has the ability to repeatedly and reversibly change some of its functions, features or behavior over time in response to changing performance requirements and variable boundary conditions, and does this with the aim of improving overall building performance."
	- building integrated PV (BIPV) and static daylighting systems are not included in CABS (not adaptive)
	- Robustness is the ability to reduce negative influences caused by external changes, it should be defined together with flexibility which describes the ability of the system to change according to the external conditions (adaptability, multi-ability and evolvability)
	- adabtable building shells can react to external changes
	- multi-ability means that one technology can perform different tasks, depending on the user or the external conditions (e.g. foldable balcony)
	- evolvability is the ability to react to changes in the long run
	- CABS are characterized by four physical categories: (Thermal, Optical, Air-flow, and Electrical)
	- time scales can range from seconds (wind changes) to seasonal changes
	- adaption can take place on a macro or a micro scale. Macro scale stands for visible, mostly mechanical changes, whereas on the micro scale mostly optical properties change
	- Control can be extrinsic (feedback loops, needs sensors, processors and actuators, either distributed or centralized) or intrinsic (direct reaction to changes e.g. by smart materials). Extrinsic control provides the opportunity of centralized control and evolvability, wheras intrinsic control utilizes less parts and a more direct feedback.  
	- all reviewed projects take into account the thermal domain, most consider wind-flows, some consider optical domain or electrical domain (mostly by PV)
	- tendency towards extrinsic control on macro scale (conservative solution with sensors and servomotors, pumps or fans as actuators)
	- CABS are still in the development phase, high risk projects. The demand for building performance simulations is increasing. 
	- Tradeoffs for CABS control: ``daylight vs. glare, views vs. privacy, fresh air vs. draught risk, solar shading vs. artificial lighting, passive solar gains vs. potential overheating". 
	- advanced supervisory control strategies are needed such as optimal control and model based predictive control with weather forecast. 
	- users should be able to override control
	- cost is still a major issue, large scale production is needed for CABS to have an impact\\



Energy efficiency of a dynamic glazing system \cite{lollini10}

	- triple glazing, shading component in outer gap and ventilation in inner gap
	- independent variables: specific fan power, venetian blind position, kind of glazing system, difference between the air gap and the indoor air temperature
	- dependent variables: U-value, g-value, energy consumption, predicted percentage of dissatisfied (PPD)(, air change, daylighting factor)\\

Parametric analysis and systems design of dynamic photovoltaic shading modules (hofer15)

	- utilization of 3D geometrical modeling to calculate mutual shading
	- parameters varied: facade orientation, distance between modules, electrical design parameters, one or two axis-tracking
	- Solar radiation calculated by use of three-component model (direct, diffuse sky and diffuse reflected radiation):

	\begin{equation}
		G_{tot,m,p}(t) = G_{dir,m,p}(t) +G_{dif,m,p}(t) + G_{ref,m,p}(t)
		\label{e:radiation}
	\end{equation}

	- Solar position generated by use of DIVA plugin
	- Shading on panels calculated as vector graphics
	- Solar insolation is averaged for every hour of the month and only one day per month is evaluated.
	- Power loss increases linearly with shaded area for 100\% lateral shading, but increases faster for smaller lateral shading. 
	- Solar altitude tracking leads to high mutual shading in summer and low mutual shading in winter, therefore seasonal variations in PV production are relatively small.
	- Overall, two axis tracking with small spacing results in highest insolation per facade area, though modules become less efficient caused by the mutual shading. --> What spacing results in highest energy production?






	
	
	
	
