%!TEX root = thesis.tex
\chapter{Outlook}
\label{ch:outlook}

While the results of this thesis are promising, further research must be done on many aspects. Optimisation algorithms must be found to determine the best state of the system with individual actuation while taking into account all influences. The optimisation algorithms could be included into the control of a real system, so that it can be at the optimum position at all times. Influences of user satisfaction and comfort have to be evaluated as well and must ultimately be included in the optimum control methods. Furthermore, the PV panels must be connected into strings for the evaluation and the influence of bypass-diodes should be included. In order to evaluate the building performance in more detail, the methodology needs to be changed to calculate building energy demand for single hours, taking into account the inertia of the system. Also the energy needed for the actuation of the panels should be modelled and included into the optimisation, to determine whether the energy savings from the improved position are higher than the actuation energy needed to get to that position. Other influences that can be evaluated in more detail include the reflectance of the panels, the ventilation, or the user interaction. In order to increase the accuracy of the simulations, an iterative optimisation algorithm could be developed and implemented in the future, in order to increase the accuracy of the simulation results. Finally, even though life cycle analysis was done before for this project, it should now be done again with the enhanced energy performance results, and could ultimately even be integrated into the simulation environment. 

%Furthermore, the electrical PV model should be enhanced to include the possibility of simulating different string connections of panels as well as the use of bypass diodes. 

%Different influences can be evaluated in more detail, such as the reflectance of the panels, the ventilation, or the user interaction. 



%Even though life cycle analysis was done before for this project, it can now be done again with the enhanced results, ultimatively even integrated into the simulation enevironment. 