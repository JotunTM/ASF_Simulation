%!TEX root = thesis.tex
\chapter{Conclusions}
\label{ch:conclusion}

%In this thesis, a simulation methodology to evaluate a dynamic photovoltaic shading system is presented, combining both electricity generation, and the energy demand of the building. It is then coupled with a post processing python script to determine the optimum system configuration for control. The methodology can be applied to evaluate different PV system geometries, building systems, building typologies and climates.

%The dynamic PV integrated shading system has clear advantages to a static system as it can adapt itself to the external environmental conditions. This enables it to orientate itself to the most energy efficient position. The optimum orientation however, strongly depends on the general efficiency of the building. Decreasing the efficiency of the heating, cooling or lighting systems will give higher preference for configurations optimised for building energy management through adaptive shading, than for PV electricity production.


%The use of LED lights, for example, reduces the weighting of the lighting energy demand. This would result in closed configurations optimised for cooling to over-ride the open positions. 

%This work ultimately presents a methodology for the planning and optimisation of sophisticated adaptive BIPV systems. %Future work will use this methodology to determine the environments and building typologies that could benefit from adaptive BIPV systems. 

The work in this thesis presents a methodology to simultaneously evaluate PV electricity production and building energy demand of photovoltaic modules as adaptive building shading systems. A parametric model was created, to easily evaluate numerous different influences and designs, and a case study was done for the adaptive solar facade project (ASF). It was possible to find the optimising angles for heating, cooling, and lighting demand as well as for PV electricity production. Furthermore, the optimising angles that minimise the overall energy use were found. When combining the building energy with the PV electricity production, it could be shown that the ASF is able to generate more electricity than is used by the building for most sunlit hours. 

The optimisation for PV electricity production was of particular importance. It was possible to show that sun-tracking does not optimise the PV power output. While sun-tracking simply follows the sun angles, the optimising algorithm yields the positions that correspond to the highest electricity output by finding the optimum mediation between maximising radiation and minimising longitudinal shading on the panels. 

A comparison of different optimisation strategies showed that the performance of the ASF is mostly influenced by the cooling demand and the PV electricity production. While an optimisation for cooling is not very beneficial for the PV electricity production, the negative effect on the cooling performance caused by an optimisation for PV is not as large. The influence of heating and lighting is smaller, because they become most dominant at hours where there is no or little sun. Nevertheless, the heating and lighting also show benefits with the overall optimisation. Of course, these results strongly depend on the building system parameters. With a lower efficiency of heating, cooling or lighting, the corresponding energy demand will become significantly more important in the optimisation and the control in general. For the chosen base case, the optimisation was able to yield energy savings of 9\% compared to a fixed solar facade at the optimum angle configuration, and the PV electricity production compensates for 41\% of the building energy demand. 

Furthermore, various parameters and their influence on the overall performance of the system were analysed. Simulations done with different building orientations showed that the ASF performs best on a south-east facing facade due to increased cooling benefits. The most electricity, however, is generated from a south facing facade. Furthermore, south-west and west facing facades perform better with PV cells laid out parallel to the upper left edge of the panel, whereas south-east and east facing facades need the cells oriented the opposite way, i.e. parallel to the upper right edge. 

Location evaluations were done for Helsinki, Zurich, Madrid and Cairo. Because of the significantly higher solar insolation and the warmer climate, the ASF performs best for the sunny regions of Madrid and Cairo. The highest PV electricity production was achieved for Madrid, this is because of the altitude angles of the sun, which are higher in Cairo and therefore generate increased self-shading on the panels. 

A sensitivity evaluation of the building system parameters showed that heating and cooling COPs strongly influence the performance of the ASF, whereas lighting and infiltration rate do not have a very strong effect on the performance. 

Evaluations of the number of simulation combinations showed that the discrete optimisation becomes more beneficial with an increasing number of evaluated combinations. Naturally the use of a smaller step size yields more accurate results of the optimisation. 

Finally, the potential of independent actuation was analysed with simplified simulations using two and three clusters of independently moving panels. It was possible to show that with two clusters, the overall benefit increased by 1\%, whereas for three clusters, it would rise up to 2.3\%. \\

In summary, the following factors may be taken into consideration for future ASF designs:

\begin{itemize}

\item PV electricity production and cooling energy demand strongly influence the ASF performance.
\item Sun-tracking does not yield the maximum electricity production, as it does not minimise longitudinal shading.
\item An ASF performs best at a south or south-east facing facade.
\item PV cells should be oriented parallel to the upper left edge for west and south-west facing facades, whereas the cells should be oriented parallel to the upper right edge for east and south-east facing facades. 
\item The energy benefit of an ASF is particularly large in warm and sunny regions, such as Madrid and Cairo.
\item With inefficient cooling systems, the energy savings of an ASF increase substantially.

\end{itemize}

This work ultimately presents a methodology for the planning and optimisation of sophisticated adaptive BIPV systems. The dynamic PV integrated shading system has clear advantages to a static system as it can adapt itself to the external environmental conditions. This enables it to orientate itself to the most energy efficient position. Through various simulations, it was possible to demonstrate and quantize the benefits of such a system, as well as suggesting factors that should be taken into account for future design considerations and further system performance evaluations. 