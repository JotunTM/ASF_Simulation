%!TEX root = thesis.tex
\chapter*{Abstract}
Building integrated photovoltaic (BIPV), dynamic shading systems and adaptive envelopes are becoming increasingly important in modern building technology. The adaptive solar facade (ASF) project represents all three of them, as it is a dynamic shading system with integrated photovoltaic (PV) cells, forming an adaptive envelope. Thus, the ASF provides a promising technology for future energy efficient buildings. 

This work presents a methodology to simultaneously calculate the building energy demand and the PV electricity production of a building with PV modules as adaptive building shading systems. A parametric model was built for dynamic evaluations and optimisations of such a system. A case study was then performed on a model representing the prototype of the ASF at the House of Natural Resources at ETH Zurich. 

It was possible to find the optimising configurations of the described system as well as the corresponding building energy demand. Furthermore, various influences were evaluated including sensitivities on the building orientation, the geographic location, the control strategy, and the building system parameters. For the chosen base case evaluation, energy benefits of 9\% were obtained when compared to a fixed solar facade at the most beneficial angle. The corresponding PV electricity output is able to compensate for 41\% of the total building energy demand. The benefits are even larger for warmer regions than Zurich, as well as for buildings that have less efficient heating and cooling systems. 
%, could be achieved through the optimization of the angle actuation. 
\newpage