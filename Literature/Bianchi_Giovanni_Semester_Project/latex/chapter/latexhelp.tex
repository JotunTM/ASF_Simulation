\chapter{Latex help}\label{KapitelBericht}

\section{Grafiken}
In \textsc{MatLab} erstellte Grafiken mussen als \verb".fig" \underline{und} \verb".eps" Datei gespeichert werden
(keine Bitmaps oder JPEGs!). Die Bilder müssen schwarzwei{\ss} sein, da graue Kurven kaum erkennbar sind. Am besten
erstellt man eine \textsc{MatLab} Skript-Datei, welche alle Bilder auf ein mal erzeugt. Die folgenden
\textsc{MatLab}-Einstellungen sind nutzlich:
\begin{verbatim}
 set(0,'DefaultTextFontName','Times');
 set(0,'DefaultAxesFontName','Times');
 set(0,'DefaultAxesFontSize',12);
 set(0,'DefaultTextFontSize',12);
 set(0,'DefaultLineLineWidth',1);
 set(0,'DefaultFigurePaperType','a4letter');
\end{verbatim}
Grafiken, welche als \verb".eps" Datei gespeichert sind, konnen in CorelDRAW als Postscript Interpreted importiert
werden und weiter mit CorelDRAW bearbeitet werden (Formeln zufügen usw.).



\section{Formeln}
Vektorgro{\ss}en schreibt man in der Regel mit kleinen fetten lateinischen Buchstaben, z.B.\ $\vx$, $\vy$, $\vz$, und
manchmal mit kleinen fetten griechischen Buchstaben, z.B.\ $\val$, $\vbe$, $\vga$. Matrixgro{\ss}en schreibt man in der
diesen Regeln ab um die übliche Notation der Klassischen Mechanik zu respektieren. Der Spinsatz, zum Beispiel, wird in
der Vorlesung Mechanik~III als
\begin{equation}\label{spinsatz}
 \dot{\vN}_S = \vTh_S\,\vPs + \vOm\times\vTh_S\,\vOm =
 \vM_S^a
\end{equation}
geschrieben. Koordinatensysteme werden geschrieben als $(O, \ve_x^I, \ve_y^I, \ve_z^I)$ und Matrizen beispielsweise als
\begin{equation}
 _K\vTh_S = \begin{pmatrix}
              A & D & E\\
              D & B & F\\
              E & F & C
            \end{pmatrix}.
\end{equation}
Formeln sind mit den Macros in \verb"def.tex" zu erstellen.