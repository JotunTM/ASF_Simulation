%!TEX root = thesis.tex

\chapter{Usage of Simulation Environment}\label{a:usage}

This appendix chapter describes the usage of the simulation environment. It provides essentially the same content as the github wiki that can be found at 

\url{https://github.com/architecture-building-systems/ASF_Simulation/wiki}.\\

First, a description on how to get the simulation environment and what tools must be installed is given. Then, the main files that run the evaluations are described in detail. Finally, a step-by-step guide leads through the detailed usage of the simulation environment. 

\section{Get ASF\_Simulation Folder}
\label{s:getFolder}
	
	In order to use the ASF simulation framework, one can either download the .zip file and unpack the folder or use the following commands within git:

	\subsection{Git Set-Up}

	In the working directory, type:

	\begin{verbatim}git init\end{verbatim}

	\noindent{Then checkout the repository:}

	\begin{verbatim}git clone https://github.com/architecture-building-systems/ASF_Simulation.git\end{verbatim}

	\noindent{To download the files, type:}

	\begin{verbatim}git pull\end{verbatim}
	
		
\section{Installation Guides}
\label{s:installation}

	This section describes, what programs and add-ons are needed and where they can be downloaded.

	\subsection{Installing Rhino}

	Rhino can be downloaded from \url{https://www.rhino3d.com/download}. At least Rhino 5 is required, an appropriate licence must be available.
	

	\subsection{Installing Grasshopper}
	Open Source add-on for Rhino, that can be found on:

	 \url{http://www.grasshopper3d.com/page/download-1}

	\subsubsection{Grasshopper Add-Ons}
	\begin{itemize}
	\item{GhPython: Enables the use of Python scripts within Grasshopper

		\url{http://www.food4rhino.com/project/ghpython?etx}}

	\item{DIVA/VIPER: Connects Grasshopper to EnergyPlus

		\url{http://diva4rhino.com/}

	 + The Zuerich-Kloten weather file must be added to \emph{C:/DIVA/WeatherData}}

	\item{Hoopsnake: For looping grasshopper scripts

		\url{http://www.food4rhino.com/project/hoopsnake?etx}}
	\item{Ladybug/Honeybee: Thermal and radiation simulations

		\url{https://github.com/mostaphaRoudsari/ladybug/blob/master/resources/Installation_Instructions.md}}
	\item{Human: Some additional functions for GH 

		\url{http://www.food4rhino.com/project/human?etx}}
	\item{Mesh Tools: Create and customise meshes within Grasshopper

		\url{http://www.food4rhino.com/project/meshedittools}}
	\end{itemize}


	\subsection{Installing Python}
	Anaconda is recommended as it is easy to create virtual environments and manage python:

	\url{https://www.continuum.io/downloads}\\

	\noindent{For manual installation, the following guide can be used:}

	\url{http://www.lowindata.com/2013/installing-scientific-python-on-mac-os-x/}\\

	\noindent{The packages numpy, scipy, matplotlib and ipython have to be included in thy python installation.}


\section{Grasshopper Simulations}

	
	Once everything described in Section \ref{s:installation} is installed, simulations can be run. This section describes the different parts of the grasshopper simulation environment, given in the \emph{main.gh} simulation script. In order to open the Grasshopper script, one can open an empty rhino file and type `grasshopper` in the command line. The \emph{main.gh} file can now be opened from the folder \emph{Simulation\_Environment} in Grasshopper.



	\subsection{General Description of \emph{main.gh}}
		GhPython scripts generate the geometry of the ASF of every possible configuration. The script then loops through every configuration and runs an EnergyPlus simulation and a Ladybug simulation on every geometry for each hourly time step. Special attention has to be given to the sections in the script which have a red frame, these sections should be checked before every simulation, to make sure that it is running correctly. Furthermore, one has to be aware of the places the results are stored, and the instructions given on how to save the data should be closely followed. 



	\subsection{Set Geometry}
	User Interaction on general geometry and simulation inputs.

	\subsubsection{ASF Simulation Inputs}
	Angles, number of clusters and the desired grid point size that will be used for the simulation are set in this section. The desired grid point size is only relevant for the Ladybug analysis.

	\subsubsection{Geometry Inputs}
	General inputs for the room and the ASF geometry. 

	\subsection{Geometry Calculations}
	Processing of the geometry inputs, creates the geometry and saves inputs.
	\subsubsection{Save Inputs}
	Python script that saves ASF geometry and simulation inputs.

	\subsubsection{Render the Building}
	GhPython script \emph{Render\_Room} creates the building geometry for the simulations. The room width, height, depth and glazing fractions of the front facade can be selected.

	\subsubsection{Render the ASF}
	\begin{itemize}
	\item{{\bf Generate Diamond Array:} Produces a matrix of coordinates of where a PV panel should exist}
	\item{{\bf Combination Maker:} Determines the combination of PV panels. When running a simulation, {\bf it must always be made sure that the right combination is connected} (either EplusComb or RadiationComb, framed in red)}
	\item{{\bf Render Diamond Array:} Generates the geometry based off the chosen combination and array.}
	\end{itemize}

	\subsection{EnergyPlus Simulation}

	In this section the EnergyPlus simulations are run.

		\subsubsection{DIVA Interface Conversion}
		\begin{itemize}
		\item{Converts ASF panels into DIVA shading elements}
		\item{Converts all interior walls into adiabatic surfaces}
		\item{Converts the front wall to a facade element}
		\item{Converts the glazed section into a window element}
		\end{itemize}

		\subsubsection{Run the Simulation}
		Run through the Viper interface. {\bf It is important that the right settings are used.} Especially the weather file is subject to change. For the DIVA analysis, this can only be done in the Viper settings. 
		\begin{itemize}
		\item{Lx set point: variable (evaluations were done with a value of 300)}
		\item{Heating COP: variable (evaluations were done with a value of 4)}
		\item{Cooling COP: variable (evaluations were done with a value of 3)}
		\item{Lighting Load: variable (evaluations were done with the default value of 11.74\,$W/m^2$}
		\item{Infiltration: variable (evaluations were done with a value of 1/h)}
		\item{Fresh Air: variable (evaluations were done with a value of 0.016 $\frac{m^3}{s\cdot person}$ )}
		\end{itemize}

		\subsubsection{Save Data}
		Python script that saves the DIVA data for post processing. The folder where it will be saved is set automatically in the script \emph{set DIVA data path} to \emph{Simulation\_Environment/data/grasshopper/DIVA/DIVA\_results}. {\bf After a simulation finishes, the acquired data has to be moved to a separate folder with a unique name, which will be used for post processing.} Alternatively, the folder \emph{DIVA\_results} can simply be renamed. It is good practice to also save a backup of the current \emph{main.gh} file to this folder ([Ctr+Alt+s] saves a backup). 

	\subsection{Loop EnergyPlus}
	Uses hoopsnake to iterate through all possible configurations. It must be made sure that the hoopsnake algorithm is connected correctly.

	\subsection{Set Weather File}
	The weather file used for the ladybug radiation analysis is set in this section. The desired .epw file must already be in the WeatherData folder and the full name of the weather file has to be input to the \emph{weatherPath} Python script. 

	\subsection{Save Details of Weather File}
	The weather file that was specified is read and relevant information on the sun position and the temperature is saved to the folder \linebreak {\it Simulation\_Environment/data/geographical\_location}. If a new weather file is used, the main.py file has to be run in \emph{initialize} mode to prepare the data for further use with grasshopper. It must be made sure that the component \emph{Ladybug\_SunPath} is enabled when a new weather-file is introduced. 


	\subsection{Loop Ladybug}
	In this section, Ladybug is looped according to the specified angle combinations and for relevant hours (see Section \ref{ss:setEvalPeriod})

	\subsection{Set Evaluation Period}
	\label{ss:setEvalPeriod}
	Analysis period is set according to the loop number and the auxiliary data on the sun positions. 

	\subsection{Ladybug Solar Analysis}
	First of all, one has to let ladybug fly. For this the ladybug\_ladybug component has to be put onto the GH screen. This component has to run first, if there still are warnings that you first have to let ladybug fly, click on the component and press CTRL+B, then disable and enable again components that show the warning. This ensures, that the ladybug\_ladybug component runs first. 

		\subsubsection{Create Mesh for Radiation Analysis}
		Creates a mesh of the ASF for the radiation analysis according to the desired grid point size. The normals of the ASF geometry are flipped so that they face away from the building. 

		\subsubsection{Create Sky Matrix}
		In the selectSkyMtx component, it is possible to either choose a specific hour of the year or a period of time, which can be chosen with the Analysis Period component. {\bf WARNING: This component is not working correctly in the current version of ladybug.} The bug has been found, fixed, and reported to the developers:

		\url{https://github.com/mostaphaRoudsari/ladybug/issues/233}

		\subsubsection{Calculate Radiance on Panels}
		Calculates the radiance on a specified geometry. The simulation is done for the chosen settings given by the SelectSkyMtx component. {\bf Toggle runIt to start the evaluation}, this can take up to 20 seconds on a fast computer. 

		\subsubsection{Sky Dome for reference}
		This component creates sky domes, that show where the radiation is coming from. 



	\subsection{Save Radiation Results}
	The detailed radiation results are saved to a .csv file with a C\# script. The results are saved to the folder \emph{Simulation\_Environment/data/grasshopper/ LadyBug/radiation\_results}. It should be made sure that this folder is empty before starting a simulation, because otherwise there might be \emph{leftover} data, which will fill up space without being used. The \emph{copyLayoutAndComb} component copies the file generated in the \emph{Geometry Calculations} section and also saves it to the radiation\_results folder for convenience. {\bf After a simulation finishes, the acquired data has to be moved to a separate folder with a unique name, which will be used for post processing.} Alternatively the \emph{radiation\_results} folder can be renamed. It is good practice to also save a backup of the current \emph{main.gh} file to this folder ([Ctr+Alt+s] saves a backup). 


\section{Python Evaluation}

	The data that was previously generated by the \emph{main.gh} script is read in by the \emph{main.py} file and post-processed to output several graphs, aggregated data, and a summary.csv file. 

	\subsection{Main Python File}
	The \emph{main.py} file is the main file for the python evaluation. There are two modes: \emph{initialize} and \emph{post\_processing}. If an evaluation is done for the first time for a specific location, it must first run in \emph{initialize} mode, otherwise it can be run directly in \emph{post\_processing} mode. 
	At the beginning of the script, there is a user interaction section. When starting an evaluation, a user must go through this section to make sure everything is set as wished. All variables are described in detail right before they are defined. 

	\subsection{Compare Results}
	The file \emph{compareResults.py} can be used to generate various plots on comparing simulation results. The corresponding result folders must be set manually, and adjustment might be needed in the script. 
	
	\subsection{Post-Process Sun Tracking}
	The file \emph{PostProcessSunTracking.py} can be used to compare a sun-tracking approach to an optimized solution. In order to run this script, the corresponding optimization results must be available and the folders that will be compared have to be manually set.

	\subsection{Create Further Visulisations}
	With the file \emph{createFiguresAfterMain.py}, additional plots can be generated that visualize the simulation results. 

	\subsection{Auxiliary Files}
	The \emph{aux\_files} folder contains all auxiliary files that are needed for the post-processing and the optimisation. It is recommended to go through an optimisation-run step-by-step, in order to clearly understand how the simulation works. 

\section{First Time Set-Up of a Simulation}
\label{s:setUp}

	All the files that are necessary to perform simulations of the ASF are in the folder Simulation\_Environment. There are two main simulation files, one for grasshopper, the other one for the python part of the simulation (\emph{main.gh}, \emph{main.py}). In order to get started with simulations, the following steps have to be taken to generate the auxiliary files that are needed:

	\begin{enumerate}
	\item Open the \emph{main.gh} file.  

	\item Assign the desired weather file that will be used for the radiation simulation in the section \emph{Set Weather File}. 

	\item Make sure that the component \emph{Ladybug\_SunPath} is enabled in the section \emph{Save details of weather file}. A new folder in {\it Simulation\_Environment/ data/geographical\_location} with the name of the location and information on temperature and the sun position will now be created. 

	\item Open the \emph{main.py} file.

	\item In the \emph{user interaction} section, the mainMode must be set to \emph{initialize} and the geoLocation must be set to the folder that was generated in grasshopper (for the zurich-kloten epw file, the corresponding folder is called \emph{Zuerich-Kloten}). The other options do not yet require any change, as they are only important for the post-processing mode.

	\item Run the \emph{main.py} script. Be aware that the first time the file is run, it will take some time, as the lookup-table for the pv-electricity-generation needs to be generated first.

	\item Once the \emph{main.py} script has finished without errors, the \emph{Ladybug\_SunPath} component in the section \emph{Save details of weather file} of the \emph{main.gh} script can be disabled again, as it is no longer needed. This will speed up the initialization of grasshopper when restarting it. 

	\end{enumerate}

	Now, simulations can be run with grasshopper, as described in the following section.

\section{Run Grasshopper Simulations}
\label{s:runGH}

	After carefully following the instructions given in Sections \ref{s:getFolder}, \ref{s:installation}, and \ref{s:setUp}, simulations can now be run using grasshopper by taking the steps described in this section.

	\subsection{General Simulation Set-Up}

	\begin{enumerate}

	\item Open \emph{main.gh}

	\item Make sure the \emph{run} switch in the \emph{E+ Simulation} section as well as the \emph{\_runIt} switch in the \emph{Ladybug Solar Analysis} section are set to False 

	\item Set the building geometry and the facade geometry for the simulation in the section \emph{Set Geometry}
	\end{enumerate}

	\subsection{Run a Ladybug Radiation Simulation}

	\begin{enumerate}

	\item Set the weather file in the section \emph{Set Weather File} (If you want to use a new weather file, follow the instruction given in Section \ref{s:setUp})

	\item Reset HoopSnake in the section \emph{Loop LadyBug}

	\item Go to the folder \emph{ASF\_Simulation/Simulation\_Environment/data/}\linebreak\emph{grasshopper/LadyBug/radiation\_results} and make sure it is empty. 

	\item Connect \emph{RadiationComb} to the \emph{combination} input of the \emph{Combination Maker V2} in the Section Geometry Calculations

	\item Test loop hoopsnake and make sure the combinations are run as desired. It can be stopped by right clicking on hoopsnake and selecting \emph{stop}. 

	\item Reset hoopsnake again. 

	\item Toggle the \emph{\_runIt} input in the \emph{Ladybug Solar Analysis} section to true. 

	\item Go again to the \emph{radiation\_results} folder and see if the .csv file for the first iteration was created. Also look at the graphical output from Ladybug in the rhino scene, make sure everything is evaluated as desired. 

	\item If everything looks fine, start to loop hoopsnake again. 

	\item  Check that the \emph{LayoutAndCombinations.txt} file was created in the \emph{radiation\_results} folder and look at the csv files to see if the results are reasonable. 

	\item  Wait until the simulation is done. This step will generally take several hours, so it is a good idea to let it run over night. 

	\item  When the simulation is over, the folder \emph{radiation\_results} has to be renamed to have a unique name, typical for the simulation, such as \emph{Radiation\_Kloten\_5x\_1y\_2clust\_SE}. 

	\item  Turn off the radiation analysis (set \emph{\_runIt} to False). This will automatically create a new and empty \emph{radiation\_results} folder.  

	\item  Save a backup of the \emph{main.gh} file ([CTR+ALT+s]) and move it to the folder where the results are saved. 
	\end{enumerate}

	\subsection{Run an EnergyPlus Building Simulation}
	\begin{enumerate}

	\item Set the desired weather file and other options in the Viper component settings in the section \emph{E+ Simulation}

	\item Set all other options in this section. 

	\item Connect \emph{EplusComb} to the \emph{combination} input of the \emph{Combination Maker V2} component in the \emph{Geometry Calculations} section.  

	\item Loop hoopsnake in the \emph{Loop E+} section while looking at the rhino scene to make sure the desired combinations will be evaluated. 

	\item Reset hoopsnake in the \emph{Loop E+} section. 

	\item Turn the run input toggle in the \emph{E+ Simulation} section to true. 

	\item Go to the folder \emph{ASF\_Simulation/Simulation\_Environment/data/\linebreak grasshopper/DIVA/DIVA\_results} and check the \emph{LayoutAndCombinations.txt} file as well as the first iteration. 

	\item Check the output of the VIPER component in the \emph{E+ Simulation} section, there should be no errors and no warnings. 

	\item When everything looks fine, loop hoopsnake in the \emph{Loop E+} section. 

	\item Check the rhino scene and the result files to make sure everything is working correctly while looping. 

	\item Rename the \emph{DIVA\_results} folder to a unique name, such as \linebreak\emph{DIVA\_Kloten\_5x\_1y\_2clust\_SE}. 

	\item Turn the \emph{run} input of VIPER to False. This will create a new, empty \emph{DIVA\_results} folder. 
\
	\item Save a backup of the \emph{main.gh} file ([CTR+ALT+s]) and move it to the folder where the results are saved. 
	\end{enumerate}


\section{Run Python}

	Once the grasshopper simulations (Section \ref{s:runGH}) are finished, the results can be post-processed in order to visualize the data, find the optimum angle combinations as well as the corresponding energies. 

	\begin{enumerate}
	\item Open \emph{main.py}

	\item Set all the post-processing options in the user interaction section (each variable is described in detail before its definition)

	\item Run \emph{main.py} (you can press F5)

	\item A folder with the corresponding optimization results will be created in \emph{ASF\_Simulation/Simulation\_Environment/results}. The created folder is named with the date and time of the simulation start. 

	\item Rename the result folder

	\item Done
	\end{enumerate}
